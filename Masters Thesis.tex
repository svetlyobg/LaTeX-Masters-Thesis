%%%%%%%%%%%%%%%
% This bachelor's/master's degree thesis style is based of already applied MS Word template in the 
% Faculty of Electrical Engineering and Electronics of Technical University Gabrovo.
% Author: Svetlozar Kosev
% Last modified: 24 June 2023
%%%%%%%%%%%%%%%
\documentclass[11pt,a4paper]{article}
%\documentclass[11pt,a4paper,twoside]{article}

\usepackage[T2A]{fontenc}
\usepackage[utf8]{inputenc}
\usepackage[bulgarian,english]{babel}

%\setmainfont{Times New Roman}

\setcounter{tocdepth}{4}        %sections + subsections in the TOC
\addto{\captionsenglish}{
	\renewcommand*{\contentsname}{Съдържание}
	\renewcommand*{\figurename}{Фигури}
	\renewcommand*{\tablename}{Таблици.}
	\renewcommand*{\refname}{Библиография}
	\renewcommand*{\appendixname}{Приложение}   
}

%Put here the other needed macro packages.
\usepackage{graphicx}           %for the raster graphics
\usepackage[warn]{textcomp}     %for the degree (\textdegree) and number (\textnumero) sign
\usepackage{hyperref}           %for the URLs
\usepackage{hyphenat}           %for hyphenation of words containing hyphen
\usepackage{fancyhdr}           %for the headers design
\usepackage[
headheight=22pt,
top    = 1.50cm,
bottom = 1.80cm,
left   = 2.50cm,
right  = 2.50cm,
includeheadfoot]{geometry} % Use similar margins to the Word Template

% Define the page styles
\fancypagestyle{titlepage}{
	\fancyhf{}
	\fancyhead[C]{\LARGE{\textbf{Технически университет – Габрово}}}
	\renewcommand{\headrulewidth}{1.5pt}
	\renewcommand{\footrulewidth}{1.5pt}
}
%\fancypagestyle{body}{
%	\fancyhf{}
%	\fancyhead[LE,RO]{Раздел \textbf{\nouppercase{\leftmark}}}
%	\fancyhead[RE,LO]{\thepage}
%}
%\fancypagestyle{contents}{
%	\fancyhf{}
%	\fancyhead[LE,RO]{\thepage}
%	\fancyhead[RE,LO]{\leftmark}
%	\renewcommand{\headrulewidth}{1.0pt}
%	\renewcommand{\footrulewidth}{1.0pt}
%}
%\fancypagestyle{introduction}{
%	\fancyhf{}
%	\fancyhead[LE,RO]{\thepage}
%	\fancyhead[RE,LO]{\leftmark}
%	\renewcommand{\headrulewidth}{1.0pt}
%	\renewcommand{\footrulewidth}{1.0pt}
%}
%\fancypagestyle{appendix}{
%	\fancyhf{}
%	\fancyhead[LE,RO]{\thepage}
%	\fancyhead[RE,LO]{Приложения}
%	\renewcommand{\headrulewidth}{1.0pt}
%	\renewcommand{\footrulewidth}{1.0pt}
%}

% Begin the actual document
\begin{document}
%	\pagestyle{body}
	
	\title{}
	\date{\vspace{-18ex}}
	\maketitle
	
	\thispagestyle{titlepage}
	\begin{center}
		{\includegraphics[width=3cm]{images/logo.jpg}\\    
			\huge{\textbf{Факултет „Електроника и електротехника”}}\\
			\huge{\textbf{Катедра „Компютърни системи и технологии”}}\\
			\vspace*{0.5cm}
			\Large{\textbf{Специалност: "Компютърни системи и технологии",}}\\
			%\Large{\textbf{Магистърска програма по "Метеорология"}}\\
			\vspace*{1.5cm}
			\Huge{\textbf{Дипломна работа}}\\
			\Large{\textbf{за}}\\
			\Large{\textbf{придобиване на образователно-квалификационна степен "магистър"}}\\
			\Large{\textbf{на}}\\
			\Large{\textbf{инж. Светлозар В. Косев, факултетен \textnumero 21605006}}\\
			\vspace*{0.7cm}
			%    \Huge{\textbf{Тема: „Изграждане и мигриране към локална сървърна инфраструктура с HPE ProLiant ML350 Gen10 и Windows Server 2016 Standard “}}\\
			\huge{\textbf{Тема: „Изграждане и мигриране към локална сървърна инфраструктура с HPE ProLiant ML350 Gen10 и Windows Server 2016 Standard “}}\\
			\vspace*{1cm}
			\begin{flushleft}                
				\large{\textbf{Ръководител катедра:}}\hspace{3cm} \large{\textbf{Научен ръководител:}}
			\end{flushleft}    
			\begin{flushright}
				% Change the \hspace argument depending of the names length.                            
				\Large{\textbf{/доц. Валентина Кукенска/}}\hspace{4cm} \Large{\textbf{/доц. Делян Генков/}}     
			\end{flushright}
			%\begin{flushleft}    
			% \Large{\textbf{Консултант:}}\\
			%\hspace{5cm}\Large{\textbf{/Eжко Бежко/}}
			%\end{flushleft}
			\vspace*{1.7cm}
			\Large{Габрово, 2021}    
		}
	\end{center}
	\newpage
	
	\tableofcontents
%	\thispagestyle{contents}
	\newpage
	
	% Hence we utilize the article document class and 'Section' higher level partitioning,
	% rather than 'report'/'Chapter', \input is more suitable than \include.
%	\clearpage
%	\thispagestyle{empty}
	%%%%%%%%%%%%%%%
% This bachelor's/master's degree thesis style is based of already applied MS Word template in the 
% Faculty of Electrical Engineering and Electronics of Technical University Gabrovo.
% Author: Svetlozar Kosev
% Last modified: 24 June 2023
%%%%%%%%%%%%%%%


%\usepackage{fontspec}
%\setmainfont{Times New Roman}

%\thispagestyle{introduction}
%\clearpage
%\thispagestyle{empty}
%\markboth{Въведение}{Introduction}
\addcontentsline{toc}{section}{Въведение}

Сървърът е софтуер/операционна система на устройство, което осигурява услуга на друга програма или потребител. В център за данни (data center), физическата машина, на която се изпълнява програма, често се нарича сървър.
		
		\begin{figure}[h]
			\centering
			\includegraphics[width=0.8\linewidth]{images/1-1-1Център за данни.png} % replace "example-image" with the filename of your image
			\caption{Център за данни}
			\label{fig:Център за данни}
		\end{figure}
		
		
В моделът клиент-сървър, сървърната програма изчаква и изпълнява заявки от клиентски програми, които могат да се изпълняват в същото време и на други компютри. Някои приложения могат да служат като клиентската програма, с изисквания за услуги, а други – като сървър на заявки от други програми.

Сървърите могат да бъдат както обикновени компютри, така и реално физически или виртуални сървъри със специфичен хардуер и софтуер. Физическия сървър е машина, която се използва за изпълнението на необходим софтуер от клиентите. В повечето случаи, виртуалният сървър е операционна система, инсталирана и конфигурирана с помощта на софтуер за виртуализация. Този тип виртуализация се нарича софтуерна виртуализация (type 2 hypervisor). Популярни софтуери за виртуализация са VirtualBox, VMware Player, KVM, vSphere, QEMU. Изискванията за конфигуриране на виртуализация са процесорът (Intel и AMD) и UEFI/BIOS-ът да я поддържат. (\textit{\textbf{mldunbound.org, 2019}})

		\begin{figure}[h]
	\centering
	\includegraphics[width=0.8\linewidth]{images/hypervisor2.png} % replace "example-image" with the filename of your image
	\caption{Тип 2 виртуализация}
	\label{fig:Тип 2 виртуализация}
\end{figure}

Другият вид виртуализация е хардуерна (type 1 hypervisor). След като се стартира физическия сървър с инсталиран виртуализатор, се показва екран, приличащ на терминал. Показват се данни за процесора, паметта, мястото за съхранение, IP и MAC адресите. Тук, най-често се използват Hyper-V, Citrix XenServer и VMware ESX.


{Въведение}\label{Introduction}
	%\input{abstract}
	\newpage
	
	\chapter{Виртуални машини}\label{Sect1}
Текст за Глава 1.
\section{1}
\section{2}
\subsubsection{2.1}{Виртуални машини}\label{Sect1}
	%\input{Section1}
	\newpage
	
	\chapter{Предишна инфраструктура, изследване и подготовка}\label{Sect2}

Текст за Глава 2.

\section{aaa}
\section{bbbb}
\subsubsection{cccc}{Предишна инфраструктура, изследване и подготовка}\label{Sect2}
	%\input{Section2}
	\newpage
	
	\chapter{Изисквания за нова инфраструктура}\label{Sect3}
Текст за Глава 3.{Изисквания за нова инфраструктура}\label{Sect3}
	%\input{Section3}
	\newpage
	
	\chapter{Проектиране на нова инфраструктура}\label{Sect4}
Текст за Глава 4.{Проектиране на нова инфраструктура}\label{Sect4}
	%\input{Section4}
	\newpage
	
	\chapter*{Речник на термини}\label{Sect5}

Текст{Речник на термини}\label{Sect5}
	%\input{Section5}
	\newpage
	


\include{abstract/Bibliography.tex}{Използвана Литература}\label{Sect7}
%\input{Section7}
\newpage

	\include{abstract/Acknowledgments.tex}\label{Ackn}

\newpage
%\section*{Благодарности}\{Ackn}
%\input{Acknowledgements}
%\newpage
%\pagestyle{contents}

%\section*{Благодарности}\{Ackn}
%\input{Acknowledgements}
%\newpage

%\pagestyle{contents}

%\input{thebiblography}
\end{document}